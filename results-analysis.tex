% Options for packages loaded elsewhere
\PassOptionsToPackage{unicode}{hyperref}
\PassOptionsToPackage{hyphens}{url}
%
\documentclass[
]{article}
\usepackage{amsmath,amssymb}
\usepackage{iftex}
\ifPDFTeX
  \usepackage[T1]{fontenc}
  \usepackage[utf8]{inputenc}
  \usepackage{textcomp} % provide euro and other symbols
\else % if luatex or xetex
  \usepackage{unicode-math} % this also loads fontspec
  \defaultfontfeatures{Scale=MatchLowercase}
  \defaultfontfeatures[\rmfamily]{Ligatures=TeX,Scale=1}
\fi
\usepackage{lmodern}
\ifPDFTeX\else
  % xetex/luatex font selection
\fi
% Use upquote if available, for straight quotes in verbatim environments
\IfFileExists{upquote.sty}{\usepackage{upquote}}{}
\IfFileExists{microtype.sty}{% use microtype if available
  \usepackage[]{microtype}
  \UseMicrotypeSet[protrusion]{basicmath} % disable protrusion for tt fonts
}{}
\makeatletter
\@ifundefined{KOMAClassName}{% if non-KOMA class
  \IfFileExists{parskip.sty}{%
    \usepackage{parskip}
  }{% else
    \setlength{\parindent}{0pt}
    \setlength{\parskip}{6pt plus 2pt minus 1pt}}
}{% if KOMA class
  \KOMAoptions{parskip=half}}
\makeatother
\usepackage{xcolor}
\usepackage[margin=1in]{geometry}
\usepackage{longtable,booktabs,array}
\usepackage{calc} % for calculating minipage widths
% Correct order of tables after \paragraph or \subparagraph
\usepackage{etoolbox}
\makeatletter
\patchcmd\longtable{\par}{\if@noskipsec\mbox{}\fi\par}{}{}
\makeatother
% Allow footnotes in longtable head/foot
\IfFileExists{footnotehyper.sty}{\usepackage{footnotehyper}}{\usepackage{footnote}}
\makesavenoteenv{longtable}
\usepackage{graphicx}
\makeatletter
\def\maxwidth{\ifdim\Gin@nat@width>\linewidth\linewidth\else\Gin@nat@width\fi}
\def\maxheight{\ifdim\Gin@nat@height>\textheight\textheight\else\Gin@nat@height\fi}
\makeatother
% Scale images if necessary, so that they will not overflow the page
% margins by default, and it is still possible to overwrite the defaults
% using explicit options in \includegraphics[width, height, ...]{}
\setkeys{Gin}{width=\maxwidth,height=\maxheight,keepaspectratio}
% Set default figure placement to htbp
\makeatletter
\def\fps@figure{htbp}
\makeatother
\setlength{\emergencystretch}{3em} % prevent overfull lines
\providecommand{\tightlist}{%
  \setlength{\itemsep}{0pt}\setlength{\parskip}{0pt}}
\setcounter{secnumdepth}{5}
\newlength{\cslhangindent}
\setlength{\cslhangindent}{1.5em}
\newlength{\csllabelwidth}
\setlength{\csllabelwidth}{3em}
\newlength{\cslentryspacingunit} % times entry-spacing
\setlength{\cslentryspacingunit}{\parskip}
\newenvironment{CSLReferences}[2] % #1 hanging-ident, #2 entry spacing
 {% don't indent paragraphs
  \setlength{\parindent}{0pt}
  % turn on hanging indent if param 1 is 1
  \ifodd #1
  \let\oldpar\par
  \def\par{\hangindent=\cslhangindent\oldpar}
  \fi
  % set entry spacing
  \setlength{\parskip}{#2\cslentryspacingunit}
 }%
 {}
\usepackage{calc}
\newcommand{\CSLBlock}[1]{#1\hfill\break}
\newcommand{\CSLLeftMargin}[1]{\parbox[t]{\csllabelwidth}{#1}}
\newcommand{\CSLRightInline}[1]{\parbox[t]{\linewidth - \csllabelwidth}{#1}\break}
\newcommand{\CSLIndent}[1]{\hspace{\cslhangindent}#1}
\ifLuaTeX
  \usepackage{selnolig}  % disable illegal ligatures
\fi
\IfFileExists{bookmark.sty}{\usepackage{bookmark}}{\usepackage{hyperref}}
\IfFileExists{xurl.sty}{\usepackage{xurl}}{} % add URL line breaks if available
\urlstyle{same}
\hypersetup{
  pdftitle={Results Simulation Study},
  pdfauthor={Florian Stijven},
  hidelinks,
  pdfcreator={LaTeX via pandoc}}

\title{Results Simulation Study}
\author{Florian Stijven}
\date{2023-10-14}

\begin{document}
\maketitle

{
\setcounter{tocdepth}{2}
\tableofcontents
}
\hypertarget{introduction}{%
\section{Introduction}\label{introduction}}

In this Rmarkdown document, we analyze the results of the simulation study for
assessing the finite sample properties of the meta TCT methods. The primary goal
of this simulation study is to asses to what degree the theoretical asymptotic
results translate to finite samples. The secondary goal is to identify finite
sample settings where the meta TCT methods are not trustworthy. The tertiary
goal is to assess the correctness of the implementation in the \texttt{TCT} R-package.

All simulated data sets were generated from multivariate normal distributions
matching the data generating model (DGM) used by Raket (2022, sec. 5.1).
This DGM represents a realistic 36-months clinical trial in
prodromal Alzheimer's disease. The control group is based on the analysis of a
selection\footnote{The following inclusing criteria were used by Raket (2022): ``at
  the baseline visit, patients must be diagnosed as having mild cognitive
  impairment, score less than or equal to 28 on the mini mental state examination
  (MMSE, range 30-0, higher scores indicate less impairment) and be amyloid
  positive according to a brain positron emission tomography scan or analysis of
  cerebrospinal fluid.''} of 556 patients from the Alzheimer's disease neuroimaging initiative
(Veitch et al. 2019).

We first present the DGM in more details together with a visualization of the
relevant mean trajectories. Second, the analysis methods we consider in this
simulation study are summarized. Finally, the simulation results are presented.

\hypertarget{data-generating-model}{%
\section{Data Generating Model}\label{data-generating-model}}

For the 556 patients selected from the Alzheimer's disease neuroimaging initiative,
the ADAS-cog scores were\footnote{13-item cognitive subscale of the Alzheimer's disease assessment scale,
  lower scores indicate less impairment (Rosen, Mohs, and Davis 1984)} available at baseline visits and 6, 12, 18, 24, and 36
months after baseline. The estimated means at the corresponding time points are
\[(19.6, 20.5, 20.9, 22.7, 23.8, 27.4)'\]
and the corresponding estimated covariance matrix is
\[\begin{pmatrix}
45.1 & 40.0 & 45.1 & 54.9 & 53.6 & 60.8 \\
40.0 & 57.8 & 54.4 & 66.3 & 64.1 & 74.7 \\
45.1 & 54.4 & 72.0 & 80.0 & 77.6 & 93.1 \\
54.9 & 66.3 & 80.0 & 109.8 & 99.3 & 121.7 \\
53.6 & 64.1 & 77.6 & 99.3 & 111.4 & 127.8 \\
60.8 & 74.7 & 93.1 & 121.7 & 127.8 & 191.4
\end{pmatrix}.\]
Patient-level data in the control group are generated from a multivariate normal
distribution with the above mean vector and covariance matrix, as in
Raket (2022, sec. 5.1). Whereas Raket (2022) considered
multiple types of treatment effects, we only consider treatment effects that
correspond to proportional slowing.

To simulate proportional slowing, we first consider the reference trajectory,
that is, the mean ADAS-cog score in the control group (\(Z = 0\)) as a function of
time since baseline (\(t\)), \[f_{0}(t; \boldsymbol{\alpha}) = E(Y_{t} | Z = 0)\]
where \(Y_t\) is the ADAScog score \(t\) months after baseline. This trajectory
should be a continuous function defined for every time point in the relevant
range, \([0, 36]\). Because we only have 6 mean estimates, at 6 distinct time
points, we interpolate between these 6 points with natural cubic interpolation.
This \emph{interpolated} reference trajectory is given in Figure
\ref{fig:data-generating-model-visualization}.

For simulating data from the treated group (\(Z = 1\)), we consider trajectories of the
following form, \[E(Y_t | Z = 1) = f_0(\gamma \cdot t; \boldsymbol{\alpha}) \]
where \(\gamma\) is the proportional slowing factor. These data are simulated from
a multivariate normal distribution with the means determined by the above
trajectory function and with the same covariance matrix as in the control group.

We consider a set of DGMs where the following elements are varied to represent
a range of realistic scenarios:

\begin{itemize}
\tightlist
\item
  \textbf{Progression Rate}. We consider a \emph{normal} and \emph{fast} progression rate. The
  normal progression rate corresponds to the mean vector presented above. For the
  fast progression rate, we change the mean vector in the control group to
  \[(18.0, 19.7, 20.9, 22.7, 24.7, 29.2)'.\] Alternatively, the fast progression
  scenario could also be interpreted as corresponding to trials where patients have
  been followed up longer.
\item
  \textbf{Treatment Effect}. We consider 3 different (proportional slowing) treatment
  effects, that is, \(\gamma \in \{1, 0.75, 0.50 \}\).
\item
  \textbf{Sample Size}. We consider 4 different total sample sizes, \(n \in \{50, 200, 500, 1000\}\).
  Note that \(n\) is the total sample size, and we assume \(1:1\) randomization in all settings.
\item
  \textbf{Duration of follow up}. We consider settings with 24 and 36 months of follow up,
  corresponding to 5 and 6 measurements of the ADAScog score, respectively.
\end{itemize}

\begin{figure}
\centering
\includegraphics{results-analysis_files/figure-latex/data-generating-model-visualization-1.pdf}
\caption{\label{fig:data-generating-model-visualization}Plot of the trajectories used in the DGMs. A slowing factor equal to 1 corresponds to no treatment effect.}
\end{figure}

\hypertarget{analysis-methods}{%
\section{Analysis Methods}\label{analysis-methods}}

In this section, two analysis methods are briefly outline. We first present the
mixed model for repeated measures (MMRM) that is fitted to the simulated data sets.
Next, we discuss the meta TCT methods which use the fitted MMRM.

\hypertarget{mixed-model-for-repeated-measures}{%
\subsection{Mixed Model for Repeated Measures}\label{mixed-model-for-repeated-measures}}

For each simulated data set, a MMRM is fitted. This is a linear mixed model
where time is treated as a categorical covariate. The \textbf{systematic part} consists
of the interaction between treatment and time, except for \(t = 0\) where we
assume that the mean outcome is equal in both treatment groups. Let \(j = 0, 1, ..., K\)
denote the measurement occasions, and \(t_j\) the corresponding months after baseline.
The mean outcome is then modeled as
\[E(Y_{t_j} | Z = z) = \beta_{z, j} \; \forall \; j \in \{ 1, ..., K \}\]
and \(E(Y_{0} | Z = 0) = E(Y_{0} | Z = 1) = \beta_0\).
This essentially
means that we have a parameter for each measurement occasion-treatment
combination, except for baseline. The \textbf{covariance matrix} is assumed to be
unstructured, but common for both treatment groups.
For all simulation scenarios, this is a correctly specified model.

These models are fitted using restricted maximum likelihood (REML) using the
\texttt{mmrm()} function from the \texttt{mmrm} R-package (Sabanes Bove et al. 2022). The parameter
estimates and corresponding variance-covariance matrix are obtained by the
\texttt{coef()} and \texttt{vcov()} methods. The hypothesis test for
\[H_0: \beta_{0, j} = \beta_{1, j} \; \forall \; j \in \{ 1, ..., K \} \]
is based on the F-test with the Kenward-Roger approximate degrees of freedom.

\hypertarget{meta-tct}{%
\subsection{Meta TCT}\label{meta-tct}}

TO DO

\hypertarget{results}{%
\section{Results}\label{results}}

In this section, the results of the simulation study are presented and analyzed.
For each setting, we consider \(5000\) replications. For estimating coverage of
95\% CIs and empirical type 1 errors, this leads to a standard error of
\(\frac{0.05*0.95}{\sqrt{5000}} = 0.003\) under nominal coverage and a nominal
type 1 error.

In this section, we focus on the generalized least squares version of meta TCT
where the fitted MMRM is re-estimated to satisfy monotonicity and range
constraints as explained in the documentation of
\texttt{TCT::constrained\_vertical\_estimator()}.

The analyses and presentation of the simulation results are, somewhat
artificially, divided into two parts. First, we present the results regarding
estimation of the acceleration factor. Second, we present the results regarding
inference. The operations characteristics of the meta TCT inference procedures
are also compared with these of MMRM.

\hypertarget{estimation}{%
\subsection{Estimation}\label{estimation}}

To evaluate estimation, we look at 3 key performance measure,

\begin{enumerate}
\def\labelenumi{\arabic{enumi}.}
\tightlist
\item
  \textbf{Bias of the estimator.} This is achieve by estimating \(E(\hat{\gamma})\)
  across different settings, and comparing the estimated expectation with the
  \emph{true value}, \(\gamma_0\).
\item
  \textbf{Mean squared error.} The mean squared error of the estimator is defined as
  \$E\{(\hat{\gamma} - \gamma\_0)\^{}2 \}. This quantity is estimated as the mean of the
  squared differences between \(\hat{\gamma}\) and \(\gamma_0\). This quantity quantifies the average
  distance between the estimator and the estimand, which depends on the variance and
  bias.
\item
  \textbf{Empirical standard deviation.} The empirical standard deviation of the estimator simply
  is the standard deviation of the estimator. This measure is estimated as the sample
  standard deviation of the estimates in each setting. This value is compared with the
  median \emph{estimated standard error}.
\end{enumerate}

In Figure @fig:expected-value-estimator, the mean estimated acceleration factor
is presented across a set of scenarios. GENERAL BIAS? BIAS in SMALL SAMPLES?

\begin{figure}
\centering
\includegraphics{results-analysis_files/figure-latex/expected-value-estimator-1.pdf}
\caption{\label{fig:expected-value-estimator}Graph of mean estimated acceleration factor across a set of simulation settings. The presented results are based on the re-estimated MMRM parameters and least-squares meta TCT. The rows correspond to different true acceleration factors while the columns correspond to the number of post-randomization measures that have been ignored in meta TCT.}
\end{figure}

In Figure (\textbf{fig:mse-estimator?}), the mean squared error of the estimator for the
common acceleration factor is presented across a set of scenarios. As expected,
the MSE decreases as a function of the sample size. Moreover, a longer follow up
and faster progression lead to a smaller MSE.

\begin{figure}
\centering
\includegraphics{results-analysis_files/figure-latex/mse-estimator-1.pdf}
\caption{\label{fig:mse-estimator}Graph of the mean squared errors of the estimator for the common accleration factor across all simulation settings. The presented results are based on the re-estimated MMRM parameters and least-squares meta TCT. The rows correspond to different true acceleration factors while the columns correspond to the number of post-randomization measures that have been ignored in meta TCT. Note that both axes are log10-transformed.}
\end{figure}

In Figure (\textbf{fig:se-estimator?}), the empirical standard deviations are plotted
together with the median estimated standard errors. For small sample size, the
standard error estimator underestimates the empirical standard error. However,
this underestimation error largely disappears for larger the larger sample sizes.

\begin{figure}
\centering
\includegraphics{results-analysis_files/figure-latex/se-estimator-1.pdf}
\caption{\label{fig:se-estimator}Graph of the empirical standard deviations and the median estimated standard errors of the estimator for the common accleration factor across a set of simulation settings. The dots, and connecting lines, represent the empirical standard deviations. The presented results are based on the re-estimated MMRM parameters and least-squares meta TCT. The rows correspond to different true acceleration factors while the columns correspond to the number of post-randomization measures that have been ignored in meta TCT. The triangles represent the median estimated standard errors.}
\end{figure}

\hypertarget{inference}{%
\subsection{Inference}\label{inference}}

To evaluate inference, we look at 2 key performance measure,

\begin{enumerate}
\def\labelenumi{\arabic{enumi}.}
\tightlist
\item
  \textbf{Type 1 error and power.} We compare the empirical type 1 error with the
  nominal rate, \(\alpha = 0.05\). We do this for both the meta TCT methods, as for
  the MMRM. The former is based on the latter, so we also expect discrepancies
  between empirical and nominal type 1 error rates for the MMRM to be reflected in
  the meta TCT methods.
\item
  \textbf{Coverage.} We asses the empirical coverage rate of the estimated \(95\%\)
  CIs.
\end{enumerate}

In Figure (\textbf{fig:error-rates-meta-tct?}), we graph the empirical type 1 error rate and
power. The corresponding operating characteristics for the F-test are superimposed
in gray.
This reveals that the type 1 error rate is inflated for \ldots{} Consequently, the power
for the corresponding settings is misleading.
For \ldots{} settings, the type 1 error is close to nominal, hence, the corresponding
empirical powers can be interpreted as such. Comparing these with the power of the
F-test reveals that the meta TCT test yields a higher power.

\begin{verbatim}
## Warning in geom_line(data = results_tbl_inference %>% filter(constraints == :
## Ignoring unknown aesthetics: shape
\end{verbatim}

\begin{figure}
\centering
\includegraphics{results-analysis_files/figure-latex/error-rates-meta-tct-1.pdf}
\caption{\label{fig:error-rates-meta-tct}Graph of empirical type 1 error rates and power across a set of simulation settings. The presented results are based on the re-estimated MMRM parameters and least-squares meta TCT. The rows correspond to different true acceleration factors while the columns correspond to the number of post-randomization measures that have been ignored in meta TCT.}
\end{figure}

In Figure (\textbf{fig:coverage-meta-tct?}), the empirical coverage rates are presented for
the same settings as before.
This reveals that there is undercoverage for \ldots{}
For \ldots{} settings, coverage is close to nominal.

\begin{figure}
\centering
\includegraphics{results-analysis_files/figure-latex/coverage-meta-tct-1.pdf}
\caption{\label{fig:coverage-meta-tct}Graph of empirical coverage across a set of simulation settings. The presented results are based on the re-estimated MMRM parameters and least-squares meta TCT. The rows correspond to different true acceleration factors while the columns correspond to the number of post-randomization measures that have been ignored in meta TCT.}
\end{figure}

\hypertarget{summary}{%
\section{Summary}\label{summary}}

The results are summarized following the goals of the simulation analysis.

Primary goal: Evaluate finite sample properties of the method, also in relation
to asymptotic inference procedures.

\begin{itemize}
\tightlist
\item
  Bias
\item
  MSE, SD, SE,
\item
  CIs and error rates
\end{itemize}

Secondary goal: Identify settings where the meta TCT methods may yield
untrustworthy results.

\begin{itemize}
\tightlist
\item
  Sample size
\item
  Duration of follow up
\item
  Discarding time points
\end{itemize}

\hypertarget{refs}{}
\begin{CSLReferences}{1}{0}
\leavevmode\vadjust pre{\hypertarget{ref-raket_progression_2022}{}}%
Raket, Lars Lau. 2022. {``Progression Models for Repeated Measures: {Estimating} Novel Treatment Effects in Progressive Diseases.''} \emph{Statistics in Medicine} 41 (28): 5537--57. \url{https://doi.org/10.1002/sim.9581}.

\leavevmode\vadjust pre{\hypertarget{ref-rosen1984new}{}}%
Rosen, Wilma G, Richard C Mohs, and Kenneth L Davis. 1984. {``A New Rating Scale for Alzheimer's Disease.''} \emph{The American Journal of Psychiatry} 141 (11): 1356--64.

\leavevmode\vadjust pre{\hypertarget{ref-mmrmpackage}{}}%
Sabanes Bove, Daniel, Julia Dedic, Doug Kelkhoff, Kevin Kunzmann, Brian Matthew Lang, Liming Li, and Ya Wang. 2022. \emph{Mmrm: Mixed Models for Repeated Measures}. \url{https://openpharma.github.io/mmrm/}.

\leavevmode\vadjust pre{\hypertarget{ref-veitch2019understanding}{}}%
Veitch, Dallas P, Michael W Weiner, Paul S Aisen, Laurel A Beckett, Nigel J Cairns, Robert C Green, Danielle Harvey, et al. 2019. {``Understanding Disease Progression and Improving Alzheimer's Disease Clinical Trials: Recent Highlights from the Alzheimer's Disease Neuroimaging Initiative.''} \emph{Alzheimer's \& Dementia} 15 (1): 106--52.

\end{CSLReferences}

\end{document}
